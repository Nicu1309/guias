\documentclass[12pt,letterpaper]{article}

\usepackage{polyglossia}
\setmainlanguage{spanish}       %Polyglossia language

\usepackage[right=2cm,left=3cm,top=2cm,bottom=2cm,headsep=0cm,footskip=0.5cm]{geometry} %Program page margins and (text)position

\usepackage{fontspec}           %Advanced font selection
\usepackage[]{unicode-math}


% It's Dangerous to Go Alone! Take This
\begin{document}
Hola

Los ambientes en \LaTeX:

\begin{center}
  Hola
\end{center}

Formateando textos:

\textbf{Hola} \textit{Hola} \texttt{Hola} \textbf{\textit{Hola}}
\\                              % Forzamos rotura de parrafo
Listas:
\begin{itemize}
  \item No numeradas
  \item Segundo elemento
  \item Tercer elemento\ldots
\end{itemize}

\begin{enumerate}
  \item Y numeradas
  \item Segundo elemento
  \item Tercer elemento\ldots
\end{enumerate}

\begin{description}
  \item[Para que?] Lista de descripciones
  \item[Itemize] Listas numeradas
  \item[Enumerate] Listas no numerada
  \item[Description] Listas descrpitivas
\end{description}

Estas listas son anidables:
\begin{itemize}
  \item Facilmente
  \begin{itemize}
    \item Tanto como desees
    \begin{enumerate}
      \item Incluso combinando listas
    \end{enumerate}
  \end{itemize}
\end{itemize}

%% Podemos saltar páginas
\clearpage
Para comenzar en la siguiente, aunque no sea muy buena idea
\end{document}
